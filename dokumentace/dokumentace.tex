\documentclass[a4paper, 11pt]{article}
\usepackage[czech]{babel}
\usepackage[top=3cm, left=2cm, text={17cm, 24cm}]{geometry}
\usepackage[utf8]{inputenc}
\usepackage{times}
\usepackage[hidelinks]{hyperref}
\usepackage[hyphenbreaks]{breakurl}
\usepackage{url}

\begin{document}

\begin{center}
\Large{Dokumentace ke 2. projektu IVS\\\medskip}
\normalsize
SOMEBODY TOUCHA MY PROJECT\\\medskip
\today\\\bigskip
\end{center}


\section*{Obsah}
\ref{uvod} \quad Úvod \hfill \pageref{uvod}\medskip \\
\ref{instalace} \quad Instalace, odinstalace programu \hfill \pageref{instalace}\medskip \\
\ref{prace} \quad Jak pracovat s programem \hfill \pageref{prace}\medskip \\
\ref{zaver} \quad Závěr \hfill \pageref{zaver}

\newpage


\section{Úvod}\label{uvod}
Program byl vytvořen v rámci školního projektu do předmětu Praktické aspekty vývoje software (IVS). Všechny informace a soubory jsou dostupné na \url{https://github.com/Selenmon/SOMEBODY-TOUCHA-MY-PROJECT---IVS}. Na projekt se vztahuje GPLv3 license, tudíž je celý projekt dostupný volně ke stažení a k úpravám. Na projekt se nevztahuje záruka kvality a správného fungování. Podrobně viz soubor LICENSE.txt.

\section{Instalace, odinstalace programu}\label{instalace}
Projekt obsahuje vygenerovaný instalátor pro instalaci na systém Ubuntu. Pro zkušené uživatele jsou na stránkách projektu \url{https://github.com/Selenmon/SOMEBODY-TOUCHA-MY-PROJECT---IVS} dostupné i zdrojové soubory a potřebný makefile pro vytvoření spustitelného souboru na jiných distribucích linuxu.

\subsection{Postup instalace na systému Ubuntu}
\begin{enumerate}
\item Instalátor lze stáhnout z odkazu \url{TODO}. Pro stažení instalačního souboru do aktuálně otevřeného adresáře v terminálu lze použít příkaz: wget \url{TODO}
\item TODO
\end{enumerate}

\section{Jak pracovat s programem}\label{prace}
TODO

\section{Závěr}\label{zaver}
Poděkování si zaslouží vedoucí vývojového týmu Jakub Spišák a ostatní členové Róbert Gajdošík, Patrik Polášek, kteří se aktivně podíleli na tvorbě a vývoji celého projektu.\par
Případné chyby lze, po přihlášení se na GitHub, nahlásit v sekci Issues.

\end{document}
